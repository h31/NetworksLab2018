\documentclass[a4paper,12pt]{extarticle}
\usepackage[utf8x]{inputenc} 
\usepackage[left=2.5cm, top=2cm, right=1cm, bottom=20mm, nohead, nofoot]{geometry}
\usepackage[T1,T2A]{fontenc}
\usepackage[russian]{babel}
\usepackage{hyperref}
\usepackage{indentfirst}
\usepackage{listings}
\usepackage{color}
\usepackage{here}
\usepackage{array}
\usepackage{multirow}
\usepackage{graphicx}
\usepackage{amsmath}
\usepackage{amssymb}
\begin{document}
	\begin{center}		% выравнивание по центру

		\large Санкт-Петербургский политехнический университет Петра Великого\\
		\large Институт компьютерных наук и технологий \\
		\large Кафедра компьютерных систем и программных технологий\\[5cm]
		% название института, затем отступ 5см
		
		\huge Сети и телекоммуникации\\[0.5cm] % название работы, затем отступ 0,5см
		\large Отчет по лабораторной работе\\[0.1cm]
		\large "Сетевые технологии"\\[4cm]

	\end{center}
	\begin{flushright} % выравнивание по правому краю
		\begin{minipage}{0.35\textwidth} % врезка в половину ширины текста
			\begin{flushleft} % выровнять её содержимое по левому краю

				\large\textbf{Работу выполнил:}\\
				\large Шайхенуров Р.Р.\\
				\large {Группа:} 43501/1\\
				
				\large \textbf{Преподаватель:}\\
				\large Алексюк А.О.\\

			\end{flushleft}
		\end{minipage}
	\end{flushright}
	
	\vfill % заполнить всё доступное ниже пространство

	\begin{center}
	\large Санкт-Петербург\\
	\large \the\year % вывести дату
	\end{center} % закончить выравнивание по центру

\thispagestyle{empty} % не нумеровать страницу
\newpage
\section{Сетевой форум}
\subsection{Цель работы}
Разработать приложение «Сетевой форум».
Задание: разработать клиент-серверную систему сетевого форума, состоящую из сервера форума и пользовательских клиентов.
Основные возможности. Серверное приложение должно реализовывать следующие функции:
\begin{enumerate}
\item Прослушивание определенного порта
\item Обработка запросов на подключение по этому порту от клиентов
\item Поддержка одновременной работы нескольких клиентов через механизм нитей
\item Регистрация подключившегося клиента
\item Выдача клиенту перечня новых сообщений («постов») форума
\item Выдача клиенту иерархического представления форума
\item Прием от клиента сообщения в ветку форума
\item Выдача списка текущих активных пользователей форума
\item Обработка запроса на отключение клиента
\item Принудительное отключение клиента
\end{enumerate}
Клиентское приложение должно реализовывать следующие функции:
\begin{enumerate}
\item Установление соединения с сервером
\item Посылка регистрационных данных клиента
\item Получение и вывод перечня новых сообщений
\item Получение и вывод иерархии форума
\item Выбор текущей ветки форума
\item Посылка сообщения в текущую ветку форума
\item Запрос текущих активных пользователей форума
\item Разрыв соединения
\item Обработка ситуации отключения клиента сервером
\end{enumerate}

\subsection{Методы обмена сообщениями}
\begin{itemize}
\item void sendErrorCode(int newsockfd, int type); - отправка кода ошибки
\item void sendSuccessCode(int newsockfd); - отправка "успешного" кода
\item void writing(int newsockfd, char *buffer); - чтение сообщения от клиента
\item void reading(int newsockfd, char *buffer); - отправка сообщения клиенту

\end{itemize}
\subsection{Формат команды}
Так как протокол выбран синхронный, то реализация команды представляет из себя текстовое сообщение, в котором содержится цифра для взаимодействия с сервером.
\begin{center}\large\textbf{[INT]}\end{center}
Обмен сообщений происходит по следующему алгоритму:
\newline
Со стороны клиента:
\begin{enumerate}
\item Прием инструкции от сервера со STATUS-кодом
\item Отправка сообщения
\end{enumerate}
Со стороны серера:
\begin{enumerate}
\item Отправка STATUS-кода клиенту
\item Отправка сообщения
\item Прием сообщения
\item Выполнения команды
\item Отправка ответа клиенту
\end{enumerate}
\subsection{Прототипы функций, реализованных на сервере:}
\begin{enumerate}
\item void actionLogin(char *buffer, int newsockfd);
\item void actionMenu(char *buffer, int newsockfd);
\item char *actionThemes(char *buffer, int newsockfd);
\item void actionMessages(char *theme, char *buffer, int newsockfd);
\item void actionNewMessage(char *buffer, int newsockfd);
\item void checkOnline(char *buffer, int newsockfd);
\item void putNewMsg(char *newMessage);
\end{enumerate}

\subsection{Описание функций:}
\begin{enumerate}
\item Авторизация клиента на форуме
\item Вывод клиенту список меню и ожидание действия
\item Вывод клиенту списка тем форума, ожидание выбора
\item Прием сообщения в ветку форума, вывод всех сообщений ветки
\item Вывод списка последних сообщений из всех тем форума
\item Вывод списка пользователей, находящихся в данный момент онлайн
\item Вставка сообщения в список последних сообщений форума
\end{enumerate}

\subsection{Описание}
Для запуска севера используется строка вида:
\begin{center}\large\textbf{[server.exe]}\end{center}
где:
\newline
\textbf{server.exe} - скомпилированый исполняемый файл сервера
\newline
Для запуска клиента используется строка вида:
\begin{center}\large\textbf{[client.exe addres port]}\end{center}
где:
\newline
\textbf{client.exe} - скомпилированый исполняемый файл клиента
\newline
\textbf{addres} - адрес на котором запущен TCP-сервер(localhost)
\newline
\textbf{port} - порт который будет прослушивать сервер(8080)
\newline
При первом запуске потребуется ввести имя пользователя и пароль в виде username\_passwd, после прохождения проверки клиент будет подключен к серверу.
Максимальное количество выводимых последних сообщений:
\begin{verbatim}
int countOfNewMsg = 3;
\end{verbatim}
Максимальная длина сообщения:
\begin{verbatim}
\#define MAX_LENGTH 256
\end{verbatim}
Используемые библиотеки сервера:
\begin{verbatim}
\#include <stdio.h>
\#include <pthread.h>
\#include <unistd.h>
\#include <cygwin/socket.h>
\#include <sys/socket.h>
\#include <strings.h>
\#include <stdlib.h>
\#include <cygwin/in.h>
\#include <memory.h>
\#include <stdbool.h>
\end{verbatim}
Тип сокета:
\begin{verbatim}
int sockfd;
sockfd = socket(AF_INET, SOCK_STREAM, 0);
\end{verbatim}

Сервер реализован с помощью enum'а, то есть имеет константные состояния для каждого клиента. Это возможно, так как используется механизм нитей, поэтому у каждого клиента своя область видимости и свое состояние сервера.
\subsection{Функции, для механизма нитей}
\begin{enumerate}
\item void *connection\_f(); - нить, подключающая клиента к серверу, выделяя ему \begin{verbatim}socketId\end{verbatim}
\item void *connect\_handler(void *args); - нить, выдающая собственную область видимости состояний сервера клиенту
\end{enumerate}
Исходный код можно посмотреть на https://github.com/Pox3R/NetworksLab2018

\section{Прикладной протокол HTTP}
\subsection{Цель работы}
Разработать «Прикладной протокол HTTP».
Задание: предоставить серверную реализацию протокола HTTP обрабатывающего следующие методы HTTP:
\begin{enumerate}
\item GET
\item POST
\item HEAD
\end{enumerate}
Клиентским приложением является браузер или программа Postman

\subsection{Методы обмена сообщениями}
Для обмена сообщениями используется браузер или программа Postman, с их помощью можно формировать запросы нужных видов и отправлять серверу
\subsection{Формат команды}
Так как протокол выбран синхронный, то реализация команды представляет из себя текстовое сообщение, в котором содержится цифра для взаимодействия с сервером.

Обмен данными происходит по следующему алгоритму:
\newline
Со стороны клиента:
Формируется HTTP запрос и отправляется на сервер
Со стороны серера:
Сервер принимает запрос, после чего обрабатывает полученное сообщение, через s.getInputStream(), после чего формирует \textbf{response header} и \textbf{response body}
и отправляет через s.getOutputStream()
\subsection{Прототипы функций, реализованных на сервере:}
\begin{enumerate}
\item public static void main(String args[]) - main-функция, в которой происходит прослушка порта и создание потока, при подключении клиента
\item public HttpServer(Socket s) - конструктор класса, запускающий поток для конкретного сокета
\item public void run() - функция, необходимая для реализации интерфейса Thread, которую будет выполнять поток
\item protected String getData(String header) - получение данных, отправленных на сервер для их разбора - обрабатывает GET запрос
\item public String parseURL(String path, String URI) - парсинг URL, необходимый для извлечения пути к файлу, к которому обратился клиент
\item protected String extract(String str, String start, String end) - функция, необходимая для извлечения URL
\item public String getPOST(String URI) - обработка POST-запроса
\item public String getHEAD(String URI) - обработка HEAD запроса
\end{enumerate}

\subsection{Описание}
Для запуска севера используется строка вида:
\begin{center}\large\textbf{[server.jar]}\end{center}
Или запустить можно в среде разработки Intellij IDEA.
\newline
Для запуска клиента - GET запрос:
\begin{enumerate}
\item Открываем браузер
\item в адресную строку вводим http://localhost:80/*имя файла, который вы хотите получить*, например, http://localhost:80/logo.jpg
\end{enumerate}
Для POST и HEAD запросов:
\begin{enumerate}
\item Открываем Postman
\item Вводим адрес, указанный выше, с файлом(Для HEAD запроса) и с Query-параметрами для POST-запроса
\item Нажимаем кнопку SEND
\item В нижней части окна получаем ответ сервера
\item В POST запросе получаем список Query-параметров и Header запроса
\item В HEAD запросе получаем Header файла, который указали
\end{enumerate}
Сервер работает асинхронно, то есть выдает то, что хочет клиент и завершает поток. Поэтому при каждом новом подключении клиента происходит пересоздание сокета, это позволяет достичь того, что поключиться к серверу смогут множество клиентов. При вводе имени файла, которого нет на сервере, сервер выдаст ошибку 404, означающую "Not Found", при попытке получить HEAD файла, не существующего на сервере, приходит HEADER с нулевыми значениями. При отправке POST-запроса, файлы не учитываются, считываются лишь его Query-параметры, которые отправляет сервер в Response Body

Используемые библиотеки сервера:
\begin{verbatim}
import java.io.*;
import java.net.*;
import java.text.DateFormat;
import java.util.Date;
import java.util.TimeZone;
\end{verbatim}

\section{Выводы}
При реализации TCP сервера возникли проблемы с обработкой файлов и строк, так как не использовались стандартные библиотеки. Написано в ручную множество функций для строк(Например, удаление из середины строки), используя стандартные CHAR'овские функции. В итоге получили Клиент-Серверное синхронное приложение: Сетевой форум, обрабатывающий по своему протоколу сообщения клиента. С помощью STATUS-кодов можно определить, что именно не так сделал клиент и легко исправить ввод. Использовалась модель конечного автомата, которая не очень хорошо подожша бы для графического приложения, но для данного случая подходит отлично. В будущем можно будет пересмотреть логику сервера, чтобы он был асинхронным, позволяющим перейти в любое меню по командам.
\newline
При реализации HTTP протокола возникли следующие сложности: обработка принятого сообщения. В итоге была написана функция, которая обрезает Request, в связи с нужным ответом клиенту. Использовались Java-socket'ы и интерфейс Thread, которые немного упростили задачу в конечном итоге. Так же сервером формируется Response в связи со стандартом RFC так, чтобы это поддерживалось браузером.
\end{document}